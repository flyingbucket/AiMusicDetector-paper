\documentclass[aspectratio=169]{beamer}
\usetheme{gotham}
\useoutertheme[
  progressbar position=foot,   % 进度条位置:foot/head/none
  progressbar style=rectangle, % 样式:rectangle/rounded box/circle
  numbering=totalframenumber   % 页码样式:none/framenumber/totalframenumber
]{gotham}

\usefonttheme{professionalfonts} % 使用系统/自定字体


% === 字体设置 ===
\usepackage[UTF8,scheme=plain,fontset=none]{ctex}
\setCJKmainfont{Source Han Serif CN}[BoldFont={Source Han Serif CN Bold}]
\setCJKsansfont{Source Han Sans CN}[BoldFont={Source Han Sans CN Bold}]
% \setCJKmonofont{Sarasa Mono CN}


% beamer 已加载 hyperref;加 unicode 以支持中文书签
\hypersetup{unicode}

% define paragraph
\providecommand{\paragraph}[1]{\smallskip\textbf{#1}\par}

% 常用包
\usepackage{amsmath,amssymb}
\usepackage{graphicx}
\graphicspath{{.}{./figs/}{./images/}{./images_in_paper/}}
\usepackage{caption}
\usepackage{subcaption}
\usepackage{float}
\usepackage{svg}
\usepackage{booktabs}
\usepackage{array}
\usepackage{threeparttable}

% 超链接(beamer 已加载 hyperref,这里只补选项)
% \hypersetup{unicode=true}

% 编号风格
\setbeamertemplate{caption}[numbered]
\setbeamertemplate{caption label separator}{.}

\title{从时域到频域:基于多分支CNN网络的AI音频检测模型}
\author{NKUMMF2025138}
\date{\today}


%---Document Begins---
\begin{document}

\begin{frame}
  \titlepage
\end{frame}


% 摘要页(无页眉,允许自动换页)
\begin{frame}[plain,allowframebreaks]{摘要}
\small

\begin{center}
  \textbf{\huge 摘\quad 要}
\end{center}
\vspace{0.2cm}

% \begin{abstract}
针对AI音频的识别问题,本文提出一种基于\textbf{多分支卷积神经网络(Multi-Branch CNN)}的
端到端 AI 音频检测与评分模型。

从\textbf{时域、频域及声学常见统计量}等多角度提取 11 类特征,分别经五个并行分支建模后融合判别。
针对单分支贡献的量化问题,引入\textbf{分支探针}机制,
基于各分支独立判别准确率确定加权系数,构建可解释的 AI 痕迹综合评分体系。

本文设计多种扰动与对抗性处理(如频谱均衡、高频注入、环境噪声混入等)评估模型鲁棒性,并结合分支贡献分析揭示其在时域包络与共振峰布局上的依赖性。实验结果表明,该模型在验证集上准确率可达 89\%–90\%,在多数轻中度扰动下保持稳定性能,综合评分在强扰动下亦具较高稳健性。本文方法具有较低计算开销与良好可扩展性,可推广至语音伪造检测、环境音识别等领域。

% \end{abstract}

\vspace{0.5em}
\noindent\textbf{关键词:} AI 音乐检测;多分支CNN;音频特征提取;探针机制;AI 痕迹评分

\end{frame}

\end{document}
