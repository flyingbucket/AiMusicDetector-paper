\documentclass[aspectratio=169]{beamer}
\usetheme{metropolis}
% \usetheme{gotham}
% \useoutertheme[
%   progressbar position=foot,   % 进度条位置:foot/head/none
%   progressbar style=rectangle, % 样式:rectangle/rounded box/circle
%   numbering=totalframenumber   % 页码样式:none/framenumber/totalframenumber
% ]{gotham}

\usefonttheme{professionalfonts} % 使用系统/自定字体


% === 字体设置 ===
\usepackage[UTF8,scheme=plain,fontset=none]{ctex}
\setCJKmainfont{Source Han Serif CN}[BoldFont={Source Han Serif CN Bold}]
\setCJKsansfont{Source Han Sans CN}[BoldFont={Source Han Sans CN Bold}]
% \setCJKmonofont{Sarasa Mono CN}


% beamer 已加载 hyperref;加 unicode 以支持中文书签
\hypersetup{unicode}

% define paragraph
\providecommand{\paragraph}[1]{\smallskip\textbf{#1}\par}

% 常用包
\usepackage{longtable,booktabs}
\usepackage{amsmath,amssymb}
\usepackage{graphicx}
\graphicspath{{.}{./figs/}{./images/}{./images_in_paper/}}
\usepackage{caption}
\usepackage{subcaption}
\usepackage{float}
\usepackage{svg}
\usepackage{booktabs}
\usepackage{array}
\usepackage{threeparttable}

% 超链接(beamer 已加载 hyperref,这里只补选项)
% \hypersetup{unicode=true}

% 编号风格
\setbeamertemplate{caption}[numbered]
\setbeamertemplate{caption label separator}{.}

\title{从时域到频域:基于多分支CNN网络的AI音频检测模型}
\author{NKUMMF2025138}
\date{\today}


%---Document Begins---
\begin{document}
\begin{frame}{摘要}
\small
\begin{itemize}
  \item 问题一:
    \begin{itemize}
      \item 基于\textbf{多分支卷积神经网络(Multi-Branch CNN)}的端到端 AI 音频检测与评分模型;
      \item 从\textbf{时域、频域及声学统计量}等角度提取 11 类特征,多分支并行融合判别;
    \end{itemize}

  \item 问题二:
    \begin{itemize}
      \item 设计\textbf{分支探针机制},依据各分支判别准确率加权,构建可解释的 AI 痕迹评分;
    \end{itemize}

  \item 问题三:
    \begin{itemize}
      \item 引入\textbf{频谱均衡、高频注入、环境噪声混入}等扰动测试模型鲁棒性;
    \end{itemize}
  \vspace{0.8cm}
  \item 模型在验证集上准确率达 \textbf{89.57\% $\pm$ 0.43\%},在多种扰动下表现稳健,具\textbf{低计算开销与良好可扩展性}。
\end{itemize}
\end{frame}
\end{document}
